%% LyX 2.3.7 created this file.  For more info, see http://www.lyx.org/.
%% Do not edit unless you really know what you are doing.
\documentclass[11pt,notitlepage]{article}
\usepackage{mathptmx}
\usepackage[latin9]{inputenc}
\usepackage{geometry}
\geometry{verbose,tmargin=1.25in,bmargin=1.25in,lmargin=1in,rmargin=1in}
\usepackage{xcolor}
\usepackage{verbatim}
\usepackage{float}
\usepackage{mathtools}
\usepackage{amsmath}
\usepackage{amssymb}
\usepackage{graphicx}
\usepackage{setspace}
\usepackage[authoryear]{natbib}
\PassOptionsToPackage{normalem}{ulem}
\usepackage{ulem}
\onehalfspacing
\usepackage[unicode=true,pdfusetitle,
 bookmarks=true,bookmarksnumbered=false,bookmarksopen=false,
 breaklinks=false,pdfborder={0 0 0},pdfborderstyle={},backref=false,colorlinks=true]
 {hyperref}

\makeatletter

%%%%%%%%%%%%%%%%%%%%%%%%%%%%%% LyX specific LaTeX commands.
%% Because html converters don't know tabularnewline
\providecommand{\tabularnewline}{\\}

%%%%%%%%%%%%%%%%%%%%%%%%%%%%%% User specified LaTeX commands.

\usepackage{amsfonts}
\usepackage{bm}
\usepackage{multicol}
\usepackage{pdflscape}
\usepackage{dcolumn}

\setcounter{MaxMatrixCols}{10}

\usepackage{fancyhdr}

\usepackage[dvipsnames]{xcolor}


% https://tex.stackexchange.com/questions/36278/box-around-theorem-statement
% https://tex.stackexchange.com/questions/581961/main-text-of-tcbtheorem-environment-absorbed-into-options
% https://xyquadrat.ch/blog/latex-boxes/
\usepackage[most]{tcolorbox}
\tcbset {
  base/.style={
    arc=0mm, 
    bottomtitle=0.5mm,
    boxrule=0mm,
    colbacktitle=black!10!white, 
    coltitle=black, 
    fonttitle=\bfseries, 
    left=2.5mm,
    leftrule=1mm,
    right=3.5mm,
    title={#1},
    toptitle=0.75mm, 
  }
}

\newtcolorbox{mainbox}[1]{
  colframe=DarkOrchid, 
  base={#1}
}
\newtcolorbox{subbox}[1]{
  colframe=black!30!white,
  base={#1}
}


%  https://www.overleaf.com/learn/latex/Paragraphs_and_new_lines
\setlength{\parindent}{0pt}

% https://stackoverflow.com/questions/4968557/latex-very-compact-itemize
\usepackage{enumitem}
\setitemize{noitemsep,topsep=0pt,parsep=0pt,partopsep=0pt}
\setenumerate{noitemsep,topsep=0pt,parsep=0pt,partopsep=0pt}

% This gives Roman section numbers
%\renewcommand{\thesection}{\Roman{section}} 
%\renewcommand{\thesubsection}{\thesection.\Alph{subsection}}

% This gives assumption numbers
\usepackage{amsmath}  
\newtheorem{assumption}{Assumption}
\newtheorem{lemma}{Lemma}

% This numbers equations by section
\numberwithin{equation}{section}

% Set link display style
\hypersetup{
    colorlinks=true,
    linkcolor={rgb:red,4;green,119;blue,204}, % Light Blue
    filecolor={rgb:red,4;green,119;blue,204}, % Light Blue
    urlcolor={rgb:red,4;green,119;blue,204}, % Light Blue
}

\@ifundefined{showcaptionsetup}{}{%
 \PassOptionsToPackage{caption=false}{subfig}}
\usepackage{subfig}
\makeatother

\begin{document}
\title{Microeconomic Analysis Lecture Notes}
\author{Arjun Srinivasan\textbf{\normalsize{}\medskip{}
}}
\maketitle

% \section{Overview\label{sec:Overview}}

Notes for the Columbia economics Ph.D. 1st-year microeconomics class (GR6211) taught by Mark Dean

\newpage{}

\tableofcontents{}

\clearpage{}

\newpage{}

\section{Choice, Preference, and Utility\label{sec:UtilityMax}}

\subsection{Introduction to representation theorems}

When dealing with models (e.g. utility maximization) that have latent (unobservable) variables, we want to find a \textbf{representation theorem}.

\begin{mainbox}{Definition: Representation theorem}
A representation theorem consists of three things:
\begin{enumerate}
    \item A data set
    \item A model
    \item A set of conditions on the data which are \textbf{necessary} and \textbf{sufficient} for it to be consistent with the model.
\end{enumerate}
\end{mainbox}

A representation theorem tells us the observable implications of a model with unobservables. Often, a representation theorem will have an associated \textbf{uniqueness result}, which tells us how precisely we have pinned down the unobservable variables.

\subsubsection{Data}

Let $X$ be a finite set of objects a person can choose from. Let $2^X$ denote the power set of $X$. 

\begin{mainbox}{Definition: Choice correspondence}
A choice correspondence tells us what a person chose from each subset of $X$. Formally, a choice correspondence $C$ is a mapping $C : 2^X \setminus \emptyset \to 2^X \setminus \emptyset$ such that $C(A) \subset A$ for all $A \in 2^X \setminus \emptyset$.
\end{mainbox}

An implicit assumption here is that the choice \emph{only depends on the elements in A}.

There are three issues with this data:
\begin{enumerate}
    \item $X$ is finite
    \item Must observe choices from \emph{all} choice sets
    \item We allow the choice of more than 1 option
\end{enumerate}

\section{Demand Functions\label{sec:Demand}}

\section{Producer Theory\label{sec:Producer}}

\section{Decision Making Under Risk\label{sec:Risk}}

\section{Decision Making Over Time\label{sec:Time}}
 
\end{document}
