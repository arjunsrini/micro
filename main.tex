%% LyX 2.3.7 created this file.  For more info, see http://www.lyx.org/.
%% Do not edit unless you really know what you are doing.
\documentclass[11pt,notitlepage]{article}
\usepackage{mathptmx}
\usepackage[latin9]{inputenc}
\usepackage{geometry}
\geometry{verbose,tmargin=1.25in,bmargin=1.25in,lmargin=1in,rmargin=1in}
\usepackage{xcolor}
\usepackage{verbatim}
\usepackage{float}
\usepackage{mathtools}
\usepackage{amsmath}
\usepackage{amsthm}
\usepackage{amssymb}
\usepackage{graphicx}
\usepackage{setspace}
\usepackage[authoryear]{natbib}
\PassOptionsToPackage{normalem}{ulem}
\usepackage{ulem}
\onehalfspacing
\usepackage[unicode=true,pdfusetitle,
 bookmarks=true,bookmarksnumbered=false,bookmarksopen=false,
 breaklinks=false,pdfborder={0 0 0},pdfborderstyle={},backref=false,colorlinks=true]
 {hyperref}
\usepackage{bbding}

\makeatletter

%%%%%%%%%%%%%%%%%%%%%%%%%%%%%% LyX specific LaTeX commands.
%% Because html converters don't know tabularnewline
\providecommand{\tabularnewline}{\\}

%%%%%%%%%%%%%%%%%%%%%%%%%%%%%% User specified LaTeX commands.

\usepackage{amsfonts}
\usepackage{bm}
\usepackage{multicol}
\usepackage{pdflscape}
\usepackage{dcolumn}

\setcounter{MaxMatrixCols}{10}

\usepackage{fancyhdr}

\usepackage[dvipsnames]{xcolor}


% https://tex.stackexchange.com/questions/36278/box-around-theorem-statement
% https://tex.stackexchange.com/questions/581961/main-text-of-tcbtheorem-environment-absorbed-into-options
% https://xyquadrat.ch/blog/latex-boxes/
\usepackage[most]{tcolorbox}
\tcbset {
  base/.style={
    arc=0mm, 
    bottomtitle=0.5mm,
    boxrule=0mm,
    colbacktitle=black!10!white, 
    coltitle=black, 
    fonttitle=\bfseries, 
    left=2.5mm,
    leftrule=1mm,
    right=3.5mm,
    title={#1},
    toptitle=0.75mm, 
  }
}

\newtcolorbox{mainbox}[1]{
  colframe=DarkOrchid, 
  base={#1}
}
\newtcolorbox{subbox}[1]{
  colframe=black!30!white,
  base={#1}
}

\newtcolorbox{thmbox}[1]{
  colframe=CornflowerBlue, 
  base={#1}
}

\DeclareMathOperator*{\argmax}{arg\,max}
\DeclareMathOperator*{\argmin}{arg\,min}
\newcommand{\defeq}{\vcentcolon=}


%  https://www.overleaf.com/learn/latex/Paragraphs_and_new_lines
\setlength{\parindent}{0pt}

% https://stackoverflow.com/questions/4968557/latex-very-compact-itemize
\usepackage{enumitem}
\setitemize{noitemsep,topsep=0pt,parsep=0pt,partopsep=0pt}
\setenumerate{noitemsep,topsep=0pt,parsep=0pt,partopsep=0pt}

% This gives Roman section numbers
%\renewcommand{\thesection}{\Roman{section}} 
%\renewcommand{\thesubsection}{\thesection.\Alph{subsection}}

% This gives assumption numbers
\theoremstyle{remark}
\newtheorem*{remark}{Remark}
\newtheorem{assumption}{Assumption}
\newtheorem{lemma}{Lemma}

% This numbers equations by section
\numberwithin{equation}{section}

% Set link display style
\hypersetup{
    colorlinks=true,
    linkcolor={rgb:red,4;green,119;blue,204}, % Light Blue
    filecolor={rgb:red,4;green,119;blue,204}, % Light Blue
    urlcolor={rgb:red,4;green,119;blue,204}, % Light Blue
}

\@ifundefined{showcaptionsetup}{}{%
 \PassOptionsToPackage{caption=false}{subfig}}
\usepackage{subfig}
\makeatother

\begin{document}
\title{Microeconomic Analysis Lecture Notes}
\author{Arjun Srinivasan\textbf{\normalsize{}\medskip{}
}}
\maketitle

% \section{Overview\label{sec:Overview}}

Notes for the Columbia economics Ph.D. 1st-year microeconomics class (GR6211) taught by Mark Dean

\newpage{}

\tableofcontents{}

\clearpage{}

\newpage{}

\section{Choice, Preference, and Utility\label{sec:UtilityMax}}

\subsection{Introduction to Representation Theorems}

When dealing with models (e.g. utility maximization) that have latent (unobservable) variables, we want to find a \textbf{representation theorem}.

\begin{mainbox}{Definition: Representation theorem}
A representation theorem consists of three things:
\begin{enumerate}
    \item A data set
    \item A model
    \item A set of conditions on the data which are \textbf{necessary} and \textbf{sufficient} for it to be consistent with the model.
\end{enumerate}
\end{mainbox}

A representation theorem tells us the observable implications of a model with unobservables. Often, a representation theorem will have an associated \textbf{uniqueness result}, which tells us how precisely we have pinned down the unobservable variables.

\subsubsection{Data}

Let $X$ be a finite set of objects a person can choose from. Let $2^X$ denote the power set of $X$. 

\begin{mainbox}{Definition: Choice correspondence}
 A choice correspondence tells us what a person chose from each subset of $X$. Formally, a choice correspondence $C$ is a mapping $C : 2^X \setminus \emptyset \to 2^X \setminus \emptyset$ such that $C(A) \subset A$ for all $A \in 2^X \setminus \emptyset$.
\end{mainbox}

An implicit assumption here is that the choice \emph{only depends on the elements in A}.

There are three issues with this data:
\begin{enumerate}
    \item $X$ is finite
    \item Must observe choices from \emph{all} choice sets
    \item We allow the choice of more than 1 option
\end{enumerate}

\subsubsection{$\alpha$, $\beta$, and Weak Axiom of Revealed Preference (WARP)}

\begin{mainbox}{Axiom $\alpha$: Independence of Irrelevant Alternatives}
If $x \in B \subseteq A$ and $x \in C(A)$, then $x \in C(B)$.
\end{mainbox}

\begin{mainbox}{Axiom $\beta$}
If $x,y \in C(A)$, $A \subseteq B$, and $y \in C(B)$, then $x \in C(B)$.
\end{mainbox}

\begin{mainbox}{Weak Axiom of Revealed Preference (WAPR)}
If $x,y \in A \cap B$, $x \in C(A)$, and $y \in C(B)$, then $x \in C(B)$.
\end{mainbox}

A dataset satisfies $\alpha$ and $\beta$ if and only if it satisfies WARP (they are equivalent).

\subsubsection{Representation Theorem}

\begin{thmbox}{Representation Theorem}
A choice correspondence on a finite $X$ has a utilty representation if and only if it satisfies axioms $\alpha$ and $\beta$.    
\end{thmbox}

\subsection{Preference Relations}

A relation $R$ on $X$ is defined as a subset of $X \times X$ where $xRy \iff (x,y) \in R \subset X \times X$.

\subsubsection{Properties of Relations}

For a relation $\succeq$ defined on $X$:

\begin{itemize}
    \item Completeness: $\forall \ x, y \in X$, either $x \succeq y$ or $y \succeq x$ (or both).
    \item Transitivity: if $x \succeq y$ and $y \succeq z$, then $x \succeq z$.
    \item Reflexivity: $\forall \ x \in X: x \succeq x$.
    \item Anti-symmetry: if $x \succeq y$ and $y \succeq x$, then $x = y$.
    \item Asymmetry: if $x \succeq y$, then $y \nsucceq x$.
    \item Symmetry: if $x \succeq y$, then $y \succeq x$.
\end{itemize}

\subsubsection{Types of Relations}

\begin{center}
\vspace{1em}
\normalsize
\begin{tabular}{l c c c c}
\textbf{Relation Type} & \textbf{Transitive} & \textbf{Reflexive} & \textbf{Anti-symmetric} & \textbf{Complete} \\ 
Preorder & {\checkmark} & {\checkmark} & &  \\
Partial Order  & {\checkmark} & {\checkmark} & {\checkmark} & \\
Linear Order & {\checkmark} & {\checkmark} & {\checkmark} &{\checkmark} \\
Preference Relation & {\checkmark} & {\checkmark} & & {\checkmark} \\
\end{tabular}
\end{center}


\begin{mainbox}{Definition: Preordered Set, Poset, Loset}
For non-empty $X$ and binary relation $R$ on $X$, $(X,R)$ is a preordered set if $R$ is a preorder, a poset if $R$ is a partial order, and loset if $R$ is a linear order.
\end{mainbox}

\subsubsection{Note on Relation Notation}

We often write $\succeq$ for a preference relation, and write the strict preference $x \succ y$ when $x \succeq y$ but not $y \succeq x$. $\succ$ is the asymmetric part $\succeq$. We also write $x \sim y$ when $x \succeq y$ and $y \succeq x$. $\sim$ is the symmetric part of $\succeq$.

\subsubsection{Order Separability, Lexicographic Preferences}

A preference relation $X$ has a utility representation if and only if it is order separable.

\begin{mainbox}{Definition: Order Separability}
A preference relation $\succeq$ on $X$ is order separable if there exists a countable sequence $e \in X^{\mathbb{N}}$ such that for every $x,y \in X$ such that $x \succ y$ there exists $e_i, e_j \in e$ such that 
$$x \succeq e_i \succ e_j \succeq y$$
\end{mainbox}

\textbf{Example: Lexicographic preferences} \\

Let $\succeq$ be a binary relation on $\mathbb{R} \times {1,2}$ such that $\{a,b\} \succeq \{c,d\} \iff (a > c) \lor ( (a = c) \land (b \geq d))$. $\succeq$ is complete, but no utility function rationalizes $\succeq$. 

\begin{proof}~
Tk.
\end{proof}

\subsubsection{Convexity}

\begin{mainbox}{Definition: Convexity}
$\succeq$ is convex if $x \succeq z$ and $y \succeq z$ implies $\forall \ \alpha \in [0,1]$ we have $\alpha x + (1 - \alpha)y \succeq z$. $\succeq$ is strictly convex if under the same conditions, $\alpha x + (1 - \alpha)y \succ z$.
\end{mainbox}

\subsubsection{Continuity}

\begin{mainbox}{Definition: Continuity (3 equivalent definitions)}
\begin{enumerate}
    \item A preference relation $\succeq$ on a metric space $X$ is continuous if for any $x, y \in X$ such that $x \succ y$ there exists an $\epsilon > 0$ such that for any $x' \in B(x, \epsilon)$ and $y' \in B(y,\epsilon)$, $x' \succ y'$.
    \item A preference relation $\succeq$ on a metric space $X$ is continuous if and only if for every $x, y \in X$ and sequence $\{ x_n, y_n\}$ such that $x_n \to x$ and $y_n \to y$, then $x_n \succeq y_n \ \forall \ n$ implies $x \succeq y$.
    \item A preference relation $\succeq$ on a metrix space $X$ is continuous if and only if, for any $x$ the sets $\{y | y \succ x\}$ and $\{y|x \succ y\}$ are open (or $\{y | y \succeq x\}$ and $\{y | x \succeq y\}$ are closed).
\end{enumerate}
\end{mainbox}

If a preference relation $\succeq$ can be represented by a continuous utility function then it is continuous.

\subsubsection{Debreu's Theorem}

Let $X$ be a separable metric space. $\succeq$ is a continuous preference relation on $X$ if and only if it can be represented by a continuous utility function. \\ 

Let $X$ be a convex subset of $\mathbb{R}^n$ and $\succeq$ be a complete preference relation on $X$. If $\succeq$ is continuous, then it can be represented by a utility function.

\subsubsection{Rationalizing Choice}

$\succeq$ rationalizes a choice correspondence $C(\cdot)$ if $C(A)= \{x \in A| x \succeq y \ \forall \ y \in A\}$. \\

$U(\cdot)$ rationalizes a choice correspondence $C(\cdot)$ if $C(A) = \underset{x \in A}{\argmax} \ U(x)$.


\subsubsection{Transitive Closures, Szpilrajn's Theorem}

\begin{mainbox}{Definition: Transitive Closure}
A transitive closure of a binary relation $R$ is a binary relation $T(R)$ that is the smallest transitive binary relation that contains $R$ (i.e. $xRy \implies xT(R)y$).
\end{mainbox}

\begin{mainbox}{Definition: Extension}
Let $\succeq$ be a preorder on $X$. An extension of $\succeq$ is a preorder $\trianglerighteq$ such that $\succeq \implies \trianglerighteq$ and $\succ \implies \triangleright$.
\end{mainbox}

\begin{thmbox}{Szpilrajn's Extension Theorem}
For any nonempty set $X$ and preorder $\succeq$ on X there exists a complete preorder that is an extension of $\succeq$.
\end{thmbox}


\subsubsection{Generalized Axiom of Revealed Preference (GARP)}

\begin{mainbox}{Revealed preference}

$x$ is directly revealed preferred $(R^D)$ to $y$ if for some choice set $A$, $y \in A$ and $x \in C(A)$. $x$ is revealed preferred $(R)$ to $y$ if we can find a set of alternatives $w_1, ..., w_n$ such that $x_1 R^D w_1 R^D ... R^D w_n R^D y$. $R$ is the transitive closure of $R^D$. \\ 

x is \textbf{strictly} revealed preferred ($S$) to y if for some choice set $A$, $y \in A$, $x \in C(A)$ and $y \notin C(A)$.


\end{mainbox}

\begin{mainbox}{Generalized Axiom of Revealed Preference (GARP)}
A choice correspondence $C$ satisfies the General Axiom of Revealed Preference (GARP) if it is never the case that $x$ is revealed preferred to y, and y is strictly revealed preferred to x.
\end{mainbox}


\begin{thmbox}{Theorem: GARP $\iff$ preference representation}
A choice correspondence $C$ on an arbitrary subset of $2^X \setminus \emptyset$ satisfies GARP if and only if it has a preference representation.
\end{thmbox}

\subsection{Random Choice}


\section{Consumer Theory\label{sec:Consumer}}

\section{Producer Theory\label{sec:Producer}}

\section{Decision Making Under Risk\label{sec:Risk}}

\section{Decision Making Over Time\label{sec:Time}}
 
\end{document}
